\input{./src/main.sty}
% Additional SI unit for Fahrenheit
\DeclareSIUnit\fahrenheit{\degree F}

\begin{document}

% Include title page
\input{./src/titlepage.tex}

\pagebreak

\section*{9.1.1.}

Show, starting from the definition of the operators for the Cartesian
components of the angular momentum, that $\hat L \times \hat L =
i\hbar \hat L$ . [Note that you should also explicitly derive any
  commutation relations that you need between the operators of the
  angular momentum components, starting from the known commutation
properties of position and (linear) momentum operators.]

\boxedanswer{
  \begin{align*}
    \bm{L} &= \bm{r} \times \bm{p} \\
    &
    \begin{aligned}
      \bm{r} &= \bm{i}x + \bm{j}y + \bm{k}z \\
      \bm{p} &= \bm{i}p_x + \bm{j}p_y +\bm{k}p_z \\
    \end{aligned} \\
    \bm{L} &=
    \begin{vmatrix}
      \bm{i} &  \bm{j} & \bm{k} \\
      x & y & z \\
      p_x & p_y & p_z \\
    \end{vmatrix} \\
    L_x &= yp_z - zp_y \\
    L_y &= zp_x - xp_z \\
    L_z &= xp_y - yp_x \\
    \hat L_x &= \hat y\hat p_z - \hat z\hat p_y \\
    \hat L_y &= \hat z\hat p_x - \hat x\hat p_z \\
    \hat L_z &= \hat x\hat p_y - \hat y\hat p_x \\
    &
    \begin{aligned}
      p_x &= -i\hbar \frac{\delta}{\delta x} \\
      p_y &= -i\hbar \frac{\delta}{\delta y} \\
      p_z &= -i\hbar \frac{\delta}{\delta z} \\
    \end{aligned} \\
    \hat L_x &= -i\hbar\left(y\frac{\delta}{\delta z} -
    z\frac{\delta}{\delta y}\right) \\
    \hat L_y &= -i\hbar\left(z\frac{\delta}{\delta x} -
    x\frac{\delta}{\delta z}\right) \\
    \hat L_z &= -i\hbar\left(x\frac{\delta}{\delta y} -
    y\frac{\delta}{\delta x}\right) \\
    \hat L \times \hat L &= \bm{i}(\hat L_x \hat L_y - \hat L_y \hat L_x) +
    \bm{j}(\hat L_y \hat L_z - \hat L_z \hat L_y) +
    \bm{k}(\hat L_z \hat L_x - \hat L_x \hat L_z) \\
    \\
    \hat L_x \hat L_y - \hat L_y \hat L_x &=
    -\hbar^2\left[\left(y\frac{\delta}{\delta z} -
      z\frac{\delta}{\delta y}\right)
      \left(z\frac{\delta}{\delta x} -
      x\frac{\delta}{\delta z}\right) -
      \left(z\frac{\delta}{\delta x} -
      x\frac{\delta}{\delta z}\right)
      \left(y\frac{\delta}{\delta z} -
      z\frac{\delta}{\delta y}\right)
    \right] \\
    \\
    \hat L_x \hat L_y - \hat L_y \hat L_x &=
    -\hbar^2\Biggl[
      \left(
        y\frac{\delta}{\delta x} + yz\frac{\delta^2}{\delta xz} -
        \cancel{yz\frac{\delta^2}{\delta z^2}} -
        \cancel{z^2\frac{\delta^2}{\delta xy}} +
        \cancel{xz\frac{\delta^2}{\delta yz}}
      \right) \\
      & \hspace{1 cm}-\left(
        \cancel{yz \frac{\delta^2}{\delta xz}} -
        xy\frac{\delta^2}{\delta z^2} -
        \cancel{z^2\frac{\delta^2}{\delta xy}} +
        x\frac{\delta}{\delta y} + \cancel{xz\frac{\delta^2}{\delta yz}}
      \right)
    \Biggr] \\
    \hat L_x \hat L_y - \hat L_y \hat L_x &=
    -\hbar^2\left[
      y\frac{\delta}{\delta x}  -
      x\frac{\delta}{\delta y}
    \right]
    \\
    \hat L_x \hat L_y - \hat L_y \hat L_x &=
    i\hbar L_z
  \end{align*}

}

\boxedanswer{

  It can be shown similarly that:

  \begin{align*}
    \hat L_y \hat L_z - \hat L_z \hat L_y &= i\hbar L_x    \\
    \hat L_z \hat L_x - \hat L_x \hat L_z &= i\hbar L_y
  \end{align*}
  So:

  \begin{align*}
    \Aboxed{\hat L \times \hat L &= \bm{i}i\hbar L_x + \bm{j}i\hbar L_y +
    \bm{k}i\hbar L_z =  i\hbar\bm{L}}
  \end{align*}
}

\pagebreak

\section*{9.2.1.}

Show explicitly in Cartesian $(x, y, z)$ coordinates that the
$\nabla^2$ and $\hat L_z$
operators commute, i.e.,

\begin{equation*}
  [ \nabla^2, \hat L_z ] = 0
\end{equation*}

\pagebreak

\section*{10.5.1.}

Consider an electron in a thin cylindrical shell potential. This
potential is zero within the
cylindrical shell and may be presumed infinite everywhere else. The
shell has inner radius $r_o$ and
thickness $L_r$, where $r_o \gg L_r$. The cylindrical shell may be
presumed to be infinite along its
cylindrical (z) axis.

\begin{enumerate}
  \item Show that the energy eigenfunctions (i.e., solutions of the
      time-independent Schr\"odingera
    equation) are, approximately,

    \begin{equation*}
      \psi(r,\phi,z) \propto \sin\frac{n\pi(r -
      r_o)}{L_r}\exp(im\phi)\exp(ik_zz)
    \end{equation*}

  \item State what the restrictions are on the values of $n$, $m$,
    and $k$ (e.g., are they real, integer, limited in
    their range?)
  \item Give an approximate expression for the corresponding energy eigenvalues.

    Note: in cylindrical polar coordinates

    \begin{equation*}
      \nabla^2 = \frac{1}{r}\frac{\delta}{\delta
      r}\left(r\frac{\delta}{\delta r}\right) +
      \frac{1}{r^2}\frac{\delta^2}{\delta\phi^2} + \frac{\delta^2}{\delta z^2}
    \end{equation*}

\end{enumerate}

\end{document}

% Additional SI unit for Fahrenheit
\DeclareSIUnit\fahrenheit{\degree F}

\begin{document}

% Include title page
\input{./src/titlepage.tex}

\pagebreak

\section*{9.1.1.}

Show, starting from the definition of the operators for the Cartesian
components of the angular momentum, that $\hat L \times \hat L =
i\hbar \hat L$ . [Note that you should also explicitly derive any
  commutation relations that you need between the operators of the
  angular momentum components, starting from the known commutation
properties of position and (linear) momentum operators.]

\boxedanswer{
  \begin{align*}
    \bm{L} &= \bm{r} \times \bm{p} \\
    &
    \begin{aligned}
      \bm{r} &= \bm{i}x + \bm{j}y + \bm{k}z \\
      \bm{p} &= \bm{i}p_x + \bm{j}p_y +\bm{k}p_z \\
    \end{aligned} \\
    \bm{L} &=
    \begin{vmatrix}
      \bm{i} &  \bm{j} & \bm{k} \\
      x & y & z \\
      p_x & p_y & p_z \\
    \end{vmatrix} \\
    L_x &= yp_z - zp_y \\
    L_y &= zp_x - xp_z \\
    L_z &= xp_y - yp_x \\
    \hat L_x &= \hat y\hat p_z - \hat z\hat p_y \\
    \hat L_y &= \hat z\hat p_x - \hat x\hat p_z \\
    \hat L_z &= \hat x\hat p_y - \hat y\hat p_x \\
    &
    \begin{aligned}
      p_x &= -i\hbar \frac{\delta}{\delta x} \\
      p_y &= -i\hbar \frac{\delta}{\delta y} \\
      p_z &= -i\hbar \frac{\delta}{\delta z} \\
    \end{aligned} \\
    \hat L_x &= -i\hbar\left(y\frac{\delta}{\delta z} -
    z\frac{\delta}{\delta y}\right) \\
    \hat L_y &= -i\hbar\left(z\frac{\delta}{\delta x} -
    x\frac{\delta}{\delta z}\right) \\
    \hat L_z &= -i\hbar\left(x\frac{\delta}{\delta y} -
    y\frac{\delta}{\delta x}\right) \\
    \hat L \times \hat L &= \bm{i}(\hat L_x \hat L_y - \hat L_y \hat L_x) +
    \bm{j}(\hat L_y \hat L_z - \hat L_z \hat L_y) +
    \bm{k}(\hat L_z \hat L_x - \hat L_x \hat L_z) \\
    \\
    \hat L_x \hat L_y - \hat L_y \hat L_x &=
    -\hbar^2\left[\left(y\frac{\delta}{\delta z} -
      z\frac{\delta}{\delta y}\right)
      \left(z\frac{\delta}{\delta x} -
      x\frac{\delta}{\delta z}\right) -
      \left(z\frac{\delta}{\delta x} -
      x\frac{\delta}{\delta z}\right)
      \left(y\frac{\delta}{\delta z} -
      z\frac{\delta}{\delta y}\right)
    \right] \\
    \\
    \hat L_x \hat L_y - \hat L_y \hat L_x &=
    -\hbar^2\Biggl[
      \left(
        y\frac{\delta}{\delta x} + yz\frac{\delta^2}{\delta xz} -
        \cancel{yz\frac{\delta^2}{\delta z^2}} -
        \cancel{z^2\frac{\delta^2}{\delta xy}} +
        \cancel{xz\frac{\delta^2}{\delta yz}}
      \right) \\
      & \hspace{1 cm}-\left(
        \cancel{yz \frac{\delta^2}{\delta xz}} -
        xy\frac{\delta^2}{\delta z^2} -
        \cancel{z^2\frac{\delta^2}{\delta xy}} +
        x\frac{\delta}{\delta y} + \cancel{xz\frac{\delta^2}{\delta yz}}
      \right)
    \Biggr] \\
    \hat L_x \hat L_y - \hat L_y \hat L_x &=
    -\hbar^2\left[
      y\frac{\delta}{\delta x}  -
      x\frac{\delta}{\delta y}
    \right]
    \\
    \hat L_x \hat L_y - \hat L_y \hat L_x &=
    i\hbar L_z
  \end{align*}

}

\boxedanswer{

  It can be shown similarly that:

  \begin{align*}
    \hat L_y \hat L_z - \hat L_z \hat L_y &= i\hbar L_x    \\
    \hat L_z \hat L_x - \hat L_x \hat L_z &= i\hbar L_y
  \end{align*}
  So:

  \begin{align*}
    \Aboxed{\hat L \times \hat L &= \bm{i}i\hbar L_x + \bm{j}i\hbar L_y +
    \bm{k}i\hbar L_z =  i\hbar\bm{L}}
  \end{align*}
}

\pagebreak

\section*{9.2.1.}

Show explicitly in Cartesian $(x, y, z)$ coordinates that the
$\nabla^2$ and $\hat L_z$
operators commute, i.e.,

\begin{equation*}
  [ \nabla^2, \hat L_z ] = 0
\end{equation*}

\pagebreak

\section*{10.5.1.}

Consider an electron in a thin cylindrical shell potential. This
potential is zero within the
cylindrical shell and may be presumed infinite everywhere else. The
shell has inner radius $r_o$ and
thickness $L_r$, where $r_o \gg L_r$. The cylindrical shell may be
presumed to be infinite along its
cylindrical (z) axis.

\begin{enumerate}
  \item Show that the energy eigenfunctions (i.e., solutions of the
      time-independent Schr\"odingera
    equation) are, approximately,

    \begin{equation*}
      \psi(r,\phi,z) \propto \sin\frac{n\pi(r -
      r_o)}{L_r}\exp(im\phi)\exp(ik_zz)
    \end{equation*}

  \item State what the restrictions are on the values of $n$, $m$,
    and $k$ (e.g., are they real, integer, limited in
    their range?)
  \item Give an approximate expression for the corresponding energy eigenvalues.

    Note: in cylindrical polar coordinates

    \begin{equation*}
      \nabla^2 = \frac{1}{r}\frac{\delta}{\delta
      r}\left(r\frac{\delta}{\delta r}\right) +
      \frac{1}{r^2}\frac{\delta^2}{\delta\phi^2} + \frac{\delta^2}{\delta z^2}
    \end{equation*}

\end{enumerate}

\end{document}

% Additional SI unit for Fahrenheit
\DeclareSIUnit\fahrenheit{\degree F}

\begin{document}

% Include title page
\input{./src/titlepage.tex}

\pagebreak

\section*{9.1.1.}

Show, starting from the definition of the operators for the Cartesian
components of the angular momentum, that $\hat L \times \hat L =
i\hbar \hat L$ . [Note that you should also explicitly derive any
  commutation relations that you need between the operators of the
  angular momentum components, starting from the known commutation
properties of position and (linear) momentum operators.]

\boxedanswer{
  \begin{align*}
    \bm{L} &= \bm{r} \times \bm{p} \\
    &
    \begin{aligned}
      \bm{r} &= \bm{i}x + \bm{j}y + \bm{k}z \\
      \bm{p} &= \bm{i}p_x + \bm{j}p_y +\bm{k}p_z \\
    \end{aligned} \\
    \bm{L} &=
    \begin{vmatrix}
      \bm{i} &  \bm{j} & \bm{k} \\
      x & y & z \\
      p_x & p_y & p_z \\
    \end{vmatrix} \\
    L_x &= yp_z - zp_y \\
    L_y &= zp_x - xp_z \\
    L_z &= xp_y - yp_x \\
    \hat L_x &= \hat y\hat p_z - \hat z\hat p_y \\
    \hat L_y &= \hat z\hat p_x - \hat x\hat p_z \\
    \hat L_z &= \hat x\hat p_y - \hat y\hat p_x \\
    &
    \begin{aligned}
      p_x &= -i\hbar \frac{\delta}{\delta x} \\
      p_y &= -i\hbar \frac{\delta}{\delta y} \\
      p_z &= -i\hbar \frac{\delta}{\delta z} \\
    \end{aligned} \\
    \hat L_x &= -i\hbar\left(y\frac{\delta}{\delta z} -
    z\frac{\delta}{\delta y}\right) \\
    \hat L_y &= -i\hbar\left(z\frac{\delta}{\delta x} -
    x\frac{\delta}{\delta z}\right) \\
    \hat L_z &= -i\hbar\left(x\frac{\delta}{\delta y} -
    y\frac{\delta}{\delta x}\right) \\
    \hat L \times \hat L &= \bm{i}(\hat L_x \hat L_y - \hat L_y \hat L_x) +
    \bm{j}(\hat L_y \hat L_z - \hat L_z \hat L_y) +
    \bm{k}(\hat L_z \hat L_x - \hat L_x \hat L_z) \\
    \\
    \hat L_x \hat L_y - \hat L_y \hat L_x &=
    -\hbar^2\left[\left(y\frac{\delta}{\delta z} -
      z\frac{\delta}{\delta y}\right)
      \left(z\frac{\delta}{\delta x} -
      x\frac{\delta}{\delta z}\right) -
      \left(z\frac{\delta}{\delta x} -
      x\frac{\delta}{\delta z}\right)
      \left(y\frac{\delta}{\delta z} -
      z\frac{\delta}{\delta y}\right)
    \right] \\
    \\
    \hat L_x \hat L_y - \hat L_y \hat L_x &=
    -\hbar^2\Biggl[
      \left(
        y\frac{\delta}{\delta x} + yz\frac{\delta^2}{\delta xz} -
        \cancel{yz\frac{\delta^2}{\delta z^2}} -
        \cancel{z^2\frac{\delta^2}{\delta xy}} +
        \cancel{xz\frac{\delta^2}{\delta yz}}
      \right) \\
      & \hspace{1 cm}-\left(
        \cancel{yz \frac{\delta^2}{\delta xz}} -
        xy\frac{\delta^2}{\delta z^2} -
        \cancel{z^2\frac{\delta^2}{\delta xy}} +
        x\frac{\delta}{\delta y} + \cancel{xz\frac{\delta^2}{\delta yz}}
      \right)
    \Biggr] \\
    \hat L_x \hat L_y - \hat L_y \hat L_x &=
    -\hbar^2\left[
      y\frac{\delta}{\delta x}  -
      x\frac{\delta}{\delta y}
    \right]
    \\
    \hat L_x \hat L_y - \hat L_y \hat L_x &=
    i\hbar L_z
  \end{align*}

}

\boxedanswer{

  It can be shown similarly that:

  \begin{align*}
    \hat L_y \hat L_z - \hat L_z \hat L_y &= i\hbar L_x    \\
    \hat L_z \hat L_x - \hat L_x \hat L_z &= i\hbar L_y
  \end{align*}
  So:

  \begin{align*}
    \Aboxed{\hat L \times \hat L &= \bm{i}i\hbar L_x + \bm{j}i\hbar L_y +
    \bm{k}i\hbar L_z =  i\hbar\bm{L}}
  \end{align*}
}

\pagebreak

\section*{9.2.1.}

Show explicitly in Cartesian $(x, y, z)$ coordinates that the
$\nabla^2$ and $\hat L_z$
operators commute, i.e.,

\begin{equation*}
  [ \nabla^2, \hat L_z ] = 0
\end{equation*}

\pagebreak

\section*{10.5.1.}

Consider an electron in a thin cylindrical shell potential. This
potential is zero within the
cylindrical shell and may be presumed infinite everywhere else. The
shell has inner radius $r_o$ and
thickness $L_r$, where $r_o \gg L_r$. The cylindrical shell may be
presumed to be infinite along its
cylindrical (z) axis.

\begin{enumerate}
  \item Show that the energy eigenfunctions (i.e., solutions of the
      time-independent Schr\"odingera
    equation) are, approximately,

    \begin{equation*}
      \psi(r,\phi,z) \propto \sin\frac{n\pi(r -
      r_o)}{L_r}\exp(im\phi)\exp(ik_zz)
    \end{equation*}

  \item State what the restrictions are on the values of $n$, $m$,
    and $k$ (e.g., are they real, integer, limited in
    their range?)
  \item Give an approximate expression for the corresponding energy eigenvalues.

    Note: in cylindrical polar coordinates

    \begin{equation*}
      \nabla^2 = \frac{1}{r}\frac{\delta}{\delta
      r}\left(r\frac{\delta}{\delta r}\right) +
      \frac{1}{r^2}\frac{\delta^2}{\delta\phi^2} + \frac{\delta^2}{\delta z^2}
    \end{equation*}

\end{enumerate}

\end{document}

% Additional SI unit for Fahrenheit
\DeclareSIUnit\fahrenheit{\degree F}

\begin{document}

% Include title page
\input{./src/titlepage.tex}

\pagebreak

\section*{9.1.1.}

Show, starting from the definition of the operators for the Cartesian
components of the angular momentum, that $\hat L \times \hat L =
i\hbar \hat L$ . [Note that you should also explicitly derive any
  commutation relations that you need between the operators of the
  angular momentum components, starting from the known commutation
properties of position and (linear) momentum operators.]

\pagebreak

\section*{9.2.1.}

Show explicitly in Cartesian $(x, y, z)$ coordinates that the
$\nabla^2$ and $\hat L_z$
operators commute, i.e.,

\begin{equation*}
  [ \nabla^2, \hat L_z ] = 0
\end{equation*}

\pagebreak

\section*{10.5.1.}

Consider an electron in a thin cylindrical shell potential. This
potential is zero within the
cylindrical shell and may be presumed infinite everywhere else. The
shell has inner radius $r_o$ and
thickness $L_r$, where $r_o \gg L_r$. The cylindrical shell may be
presumed to be infinite along its
cylindrical (z) axis.

\begin{enumerate}
  \item Show that the energy eigenfunctions (i.e., solutions of the
      time-independent Schr\"odingera
    equation) are, approximately,

    \begin{equation*}
      \psi(r,\phi,z) \propto \sin\frac{n\pi(r -
      r_o)}{L_r}\exp(im\phi)\exp(ik_zz)
    \end{equation*}

  \item State what the restrictions are on the values of $n$, $m$,
    and $k$ (e.g., are they real, integer, limited in
    their range?)
  \item Give an approximate expression for the corresponding energy eigenvalues.

    Note: in cylindrical polar coordinates

    \begin{equation*}
      \nabla^2 = \frac{1}{r}\frac{\delta}{\delta
      r}\left(r\frac{\delta}{\delta r}\right) +
      \frac{1}{r^2}\frac{\delta^2}{\delta\phi^2} + \frac{\delta^2}{\delta z^2}
    \end{equation*}

\end{enumerate}

\end{document}
